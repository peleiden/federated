% !TeX spellcheck = en_GB
% !BIB TS-program = biber
\documentclass[12pt, fleqn]{article}

\usepackage[utf8]{inputenc}
\usepackage[english]{babel}

\PassOptionsToPackage{hyphens}{url}\usepackage{hyperref}
\usepackage[top=2.5cm, bottom=2.5cm, left=3cm, right=3cm, includeheadfoot]{geometry}
\usepackage{fancyhdr}
\usepackage{graphicx}
\usepackage{float}
\usepackage{changepage}
\usepackage[nottoc, numbib]{tocbibind}
\usepackage{lastpage}
\usepackage{setspace}
\usepackage[bottom]{footmisc}
\usepackage{float}

\usepackage{amsmath}
\usepackage{amssymb}
\usepackage{nicefrac}
\usepackage{icomma}

\usepackage{color} %red, green, blue, yellow, cyan, magenta, black, white
\usepackage[dvipsnames]{xcolor}
\usepackage{titlesec}
\usepackage{listings}
\usepackage{multirow}

\usepackage{textcomp} % To avoid annoying \perthousand, \micro warnings
\usepackage{subfiles}
\usepackage{csquotes} % To avoid biber warnings
\usepackage[backend=biber, style=alphabetic, citestyle=alphabetic, maxcitenames=4, maxbibnames=4, mincitenames=2]{biblatex}
\usepackage{tabularx}

\usepackage{pgf,tikz}
\usetikzlibrary{arrows}
\usetikzlibrary{decorations.markings}

\usepackage{enumitem}
\setlist{noitemsep}

\allowdisplaybreaks
% Nicer chapters
\titleformat{\chapter}{\normalfont\huge}{\thechapter.}{20pt}{\huge}
\titlespacing*{\chapter}{0pt}{0pt}{40pt}

\fancypagestyle{plain}
{
    \fancyhf{}
    \rfoot{Page \thepage{}~of  \pageref{LastPage}}
    \renewcommand{\headrulewidth}{0pt}
}
\pagestyle{fancy}
\fancyhf{}

\lstset{frame=None,
    language={python},
    inputencoding=ansinew,
    literate=
    {æ}{{\ae}}1
    {å}{{\aa}}1
    {ø}{{\o}}1
    {Æ}{{\AE}}1
    {Å}{{\AA}}1
    {Ø}{{\O}}1,
    aboveskip=3mm,
    belowskip=3mm,
    showstringspaces=false,
    columns=flexible,
    basicstyle={\footnotesize\ttfamily},
    numbers=left,
    numberstyle=\tiny\color{gray}\ttfamily,
    keywordstyle=\color{blue}\ttfamily,
    ndkeywordstyle=\color{blue}\ttfamily,
    commentstyle=\color{gray}\ttfamily,
    stringstyle=\color{OliveGreen}\ttfamily,
    breaklines=true,
    breakatwhitespace=true,
    tabsize=4,
    escapeinside={<@}{@>},
    lineskip={-1.5pt},
    xleftmargin=1cm,
    xrightmargin=1cm
}

% Helper commands
\newcommand{\code}[1]{{\texttt{\small#1}}}
\newcommand{\numberthis}{\addtocounter{equation}{1}\tag{\theequation}}
\newcommand{\acomm}[1]{\hspace{2.5cm}\text{#1}}
\newcommand{\low}[1]{\ensuremath{_\textup{#1}}}

\newcommand{\andim}{\textup{ and }}
\newcommand{\raq}{\Rightarrow\quad}
\newcommand{\lraq}{\Leftrightarrow\quad}
\newcommand{\qandq}{\quad\wedge\quad}
\newcommand{\qorq}{\quad\vee\quad}
\newcommand{\diff}[2]{\ensuremath{\frac{\md #1}{\md #2}}}
\newcommand{\md}{\ensuremath{\text{d}}}

\newcommand{\ctp}[1]{\ensuremath{\cdot10^{#1}}}
\newcommand{\reci}{\ensuremath{^{-1}}}
\newcommand{\twopow}{\ensuremath{^{2}}}
\newcommand{\re}[1]{\ensuremath{^{#1}}}

\newcommand{\me}{\ensuremath{\operatorname{e}}}
\newcommand{\eul}[1]{\ensuremath{\me^{#1}}}
\newcommand{\len}[1]{\ensuremath{\left\lvert#1\right\rvert}}
\newcommand{\half}{\ensuremath{\frac{1}{2}}}
\newcommand{\third}{\ensuremath{\frac{1}{3}}}
\newcommand{\fourth}{\ensuremath{\frac{1}{4}}}
\newcommand{\transpose}[1]{\ensuremath{#1^{\textup T}}}

\newcommand{\NN}{\ensuremath{\mathbb N}}
\newcommand{\ZZ}{\ensuremath{\mathbb Z}}
\newcommand{\QQ}{\ensuremath{\mathbb Q}}
\newcommand{\RR}{\ensuremath{\mathbb R}}
\newcommand{\CC}{\ensuremath{\mathbb C}}
\newcommand{\LL}{\ensuremath{\mathbb L}}
\newcommand{\PP}{\ensuremath{\mathbb P}}

\newcommand{\unit}[1]{\ensuremath{\:\text{#1}}}
\newcommand{\pro}{\ensuremath{\unit{\%{}}}}

%Kommandoer til ændring af ligestillingsmargner
\newcommand{\jl}[1]{\multicolumn{1}{l}{#1}}
\newcommand{\jc}[1]{\multicolumn{1}{c}{#1}}
\newcommand{\jr}[1]{\multicolumn{1}{r}{#1}}
\newcommand{\jls}[1]{\multicolumn{1}{l|}{#1}}
\newcommand{\jcs}[1]{\multicolumn{1}{c|}{#1}}
\newcommand{\jrs}[1]{\multicolumn{1}{r|}{#1}}



\addbibresource{references.bib}

\chead{}
\rhead{Technical University of Denmark}
\rfoot{Page \thepage{}~of \pageref{LastPage}}

\graphicspath{{imgs/}}
\linespread{1.15}
\newcommand{\abbrv}[2]{\vspace{0.1cm}\textbf{#1} & #2\\}

\begin{document}

\begin{titlepage}
    \centering
    {\huge\bfseries Eating Delicious Raspberry Pie\par}
    \vspace{1.5cm}
    \includegraphics[width=0.15\linewidth]{dtu-logo.pdf}\\[1ex]
    {\scshape\Large Technical University of Denmark \par}
    \vspace{.5cm}
    \begin{large}
        \centering
        \begin{tabular}{ccc}
            Søren Winkel Holm & Asger Laurits Schultz & Gustav Lang Moesmand\\
            \code{s183911@dtu.dk} & \code{s183912@dtu.dk} & \code{s174169@dtu.dk}
        \end{tabular}
    \end{large}\par
    \vfill
    \begin{abstract}
        Federated Learning (FL) is emerging as a crucial mechanism for assuring user privacy in large-scale machine learning.
        In this project, we investigate FL compared to centralized learning on CIFAR-10 and MNIST.
        Focusing on practical challenges of FL, we develop code for managing a federation of 20 Raspberry Pi's reachable from anywhere as a simulation of real-world devices.
        On this setup, we implement the foundational FL algorithm, FedAvg, and test its final performance and convergence time efficiency when varying the number of devices available and the amount of local work performed on each device.
        Robustness is tested by introducing non-iid. device data distributions and adding image noise on a subset of clients.
        Finally, we investigate how these robustness tests are handled by the more flexible aggregation scheme, FedDF, employing ensemble distillation.
    \end{abstract}
    \vfill
    Project in 02460 Advanced Machine Learning\par
    Supervisor:\par
    Bo Li, DTU Compute
    \vfill
    {\large May 31, 2022\par}
\end{titlepage}

\tableofcontents

\setlength{\headheight}{15pt}
\addtolength{\topmargin}{-2.5pt}.

\section{Introduction}
\label{sec:intro}
\begin{enumerate}
    \item Motivation
    \item Our approach
    \item Literature
\end{enumerate}


\section{Methods}%
\label{sec:methods}
\subsection{Physical Devices}
\begin{enumerate}
    \item Single server HPC 
    \item 20 physical Raspberry Pi's
    \item Networking setup
\end{enumerate}

\subsection{FL Methods}
\begin{enumerate}
    \item FedAvg and all parameters
    \item FedDF
\end{enumerate}

\subsection{Data Imbalance and Noise}
\begin{enumerate}
    \item Dirichlet
    \item Noise approach
\end{enumerate}

\subsection{Deep Learning Problem}
\begin{enumerate}
    \item Architecture
    \item Optimization
    \item Datasets
\end{enumerate}

\subsection{Evaluation}
\begin{enumerate}
    \item Introduce types of experiments
\end{enumerate}

\section{Results}%
\label{sec:results}

\section{Discussion}%
\label{sec:discussion}


\renewcommand*{\bibfont}{\normalfont\footnotesize}
\lhead{Bibliography}
\printbibliography[heading=bibintoc]

\appendix
\lhead{Appendix}

\end{document}
